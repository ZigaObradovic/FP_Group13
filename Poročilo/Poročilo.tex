\documentclass[11pt,a4paper]{article}

\usepackage[utf8]{inputenc}
\usepackage[slovene]{babel}
\usepackage{amsmath,amssymb}
\usepackage{graphicx}
\usepackage{geometry}
\usepackage{booktabs}
\usepackage{hyperref}
\geometry{margin=2.5cm}

\title{Subpath number v kubičnih grafih:\\
minimizacija in protimeri za grafe $L_n$}
\author{Skupina 13: Subpath number -- minimalni kubični grafi}
\date{\today}

\begin{document}
\maketitle

\section{Uvod}

V projektu obravnavamo problem minimizacije števila poti v razredu kubičnih grafov,
torej grafov, kjer ima vsako vozlišče stopnjo $3$.
Naj bo $G=(V,E)$ povezan graf. 
\emph{Subpath number} grafa $G$, označen s $pn(G)$, definiramo kot število vseh
preprostih poti v grafu, vključno s trivialnimi potmi dolžine $0$.

Preprosta pot dolžine $\ell$ je zaporedje vozlišč
\[
(v_0, v_1, \dots, v_\ell),
\]
kjer se vozlišča ne ponavljajo ($v_i \neq v_j$ za vse $0 \le i < j \le \ell$),
vsak par zaporednih vozlišč pa je povezan z robom grafa 
(za vsak $1 \le i \le \ell$ je $v_{i-1}v_i \in E$).
Ker je graf neorientiran, poti, ki gredo v nasprotni smeri, štejemo ločeno.

V literaturi je bila postavljena domneva, da za vsako sodo število vozlišč
$n \ge 10$ obstaja natanko en kubični graf $L_n$, ki minimizira subpath number
med vsemi kubičnimi grafi na $n$ vozliščih.
Znano je, da domneva za dovolj velika $n$ ne drži, ni pa znano,
pri katerem najmanjšem $n$ se pojavi prva protimera.

Cilji projekta so bili:
\begin{itemize}
  \item izčrpno preveriti domnevo za manjša $n$, kjer je število kubičnih grafov
  še obvladljivo;
  \item za večja $n$ uporabiti metahevristiko \emph{simulated annealing} in poiskati
  grafe z manjšim $pn(G)$ od $pn(L_n)$;
  \item konstruirati nove družine grafov (z uporabo gradnikov iz definicije $L_n$)
  in primerjati njihov subpath number z $pn(L_n)$.
\end{itemize}

\section{Teoretično ozadje}

\subsection{Subpath number}

Za poljuben povezan graf $G=(V,E)$ definiramo
\[
pn(G) = \#\{\text{preprostih poti v } G\},
\]
pri čemer kot poti štejemo tudi trivialne poti dolžine $0$
(vsako vozlišče samo zase).
Subpath number je torej globalna mera, ki upošteva vse možne preproste poti,
ne le najkrajših ali najdaljših.

Že za zelo majhne grafe se $pn(G)$ hitro poveča.
Na primer:
en sam rob ima $pn(G)=4$ (dve trivialni poti in dve usmerjeni poti po robu),
trikotnik ima $pn(G)=15$, kvadrat $pn(G)=28$, popoln graf $K_4$ pa $pn(G)=64$.
To nakazuje, da bo izračun $pn(G)$ v splošnem eksponenten v številu vozlišč
in je zato primeren predvsem za grafe z zmernim $n$.

\subsection{Družina grafov $L_n$ in domneva}

Grafi $L_n$ so definirani za soda $n \ge 10$ in so zgrajeni iz več kopij grafa
$K_4 - e$ (popoln graf na štirih vozliščih, iz katerega odstranimo en rob),
ki so povezane v linearno verigo. Na koncih verige so dodani t.\ i.\ pendant bloki.

Natančneje:
\begin{itemize}
  \item če je $n = 4q - 2$, je $L_n$ sestavljen iz $(n-10)/4$ kopij grafa $K_4 - e$, povezanih v verigo, na obeh koncih pa sta pendant bloka na $5$ vozlišč;
  \item če je $n = 4q$, je $L_n$ sestavljen iz $(n-12)/4$ kopij grafa $K_4 - e$, na koncih pa sta pendant bloka na $5$ in $7$ vozlišč.
\end{itemize}

Geometrijsko imajo grafi $L_n$ podobo verige blokov, ki se zaključuje z dvema 
malce večjima komponentama. Na zaslonu zvezka smo za ilustracijo narisali grafe
$L_n$ za $n=10,12,14,16,18,20,22,24$; pri majhnih $n$ je veriga kratka,
za $n=24$ pa že vidimo izrazito ``kačasto'' strukturo.

Domneva, ki jo preverjamo, je:

\medskip
\noindent
\textbf{Domneva 10.}
\emph{V razredu kubičnih grafov na $n$ vozliščih, kjer je $n$ sodo,
je graf $L_n$ edini graf, ki minimizira subpath number.}
\medskip

Ker je bilo teoretično dokazano, da domneva za dovolj velika $n$ ne drži,
je glavno odprto vprašanje, pri katerih najmanjših $n$ se $L_n$ neha obnašati
kot minimizator.

\section{Metodologija}

\subsection{Konstrukcija družin grafov}

Pri implementaciji ne uporabljamo formalne definicije $L_n$ neposredno,
temveč ga zgradimo iz treh gradnikov, ki izhajajo iz grafa $K_4-e$:

\begin{itemize}
  \item \textbf{levi gradnik} je graf $K_4-e$, na katerega dodamo novo vozlišče,
  povezano z obema vozliščema stopnje $2$; dobimo pendant blok na $5$ vozliščih
  z naravno ``priklopno'' točko;
  \item \textbf{srednji gradnik} je graf $K_4-e$ z označenima dvema vozliščema
  stopnje $2$, ki služita kot levi in desni priklopni točki v verigi;
  \item \textbf{desni gradnik} je razširitev grafa $K_4-e$ z dvema dodatnima
  vozliščema in zgornjim vozliščem, kar skupaj tvori pendant blok na $7$ vozliščih.
\end{itemize}

Graf $L_n$ dobimo tako, da:
\begin{enumerate}
  \item začnemo z levim gradnikom;
  \item dodamo ustrezno število srednjih gradnikov in jih povezujemo v verigo
  prek označenih vozlišč stopnje $2$;
  \item na koncu verigo zaključimo bodisi z levim bodisi z desnim gradnikom,
  tako da ima graf natanko $n$ vozlišč in ostane kubičen.
\end{enumerate}

Podobno smo definirali dve novi družini grafov, namenjeni iskanju protimerov:

\begin{description}
  \item[star1 (funkcija \emph{build\_star})] grafe zgradimo tako, da na kratko
  ``hrbtenico'' vozlišč stopnje $3$ pripenjamo več levi h gradnikov.
  Dobljena struktura ima eno osrednje vozlišče in več ``krakov'', na katerih
  so pendant bloki. Vizualno so ti grafi podobni zvezdi z več kraki.
  \item[star2 (funkcija \emph{build\_star2})] uporabi podobno shemo, vendar
  hrbtenico po potrebi podaljša še z vmesnim (srednjim) gradnikom.
  Ta družina je bolje prilagojena določenim vrednostim $n$ (modularno po $6$).
\end{description}

V zvezku smo narisali grafe obeh družin za npr.\ $n=10,12,\dots,32$.
Pri $L_n$ se vidi skoraj čista veriga, pri zvezdastih grafih pa izrazit
osrednji vozlišči in trije ali več krakov, kar intuitivno zmanjšuje število
možnih dolgih poti.

\subsection{Izračun subpath number}

Za izračun $pn(G)$ potrebujemo način, kako sistematično našteti vse
preproste poti v grafu, ne da bi kakšno izpustili ali prešteli dvakrat.

Uporabimo naslednjo idejo:

\begin{enumerate}
  \item Vsakemu vozlišču grafa priredimo indeks od $0$ do $n-1$,
  da lahko množico obiskanih vozlišč predstavimo z bitno masko.
  \item Za vsako vozlišče $v$ zaženemo globinsko iskanje (DFS) po grafu,
  pri čemer maska označuje, katera vozlišča so že na trenutni poti.
  \item Ob vsakem rekurzivnem klicu štejemo trenutno pot kot eno
  (tudi pot dolžine $0$, kjer je na poti samo začetno vozlišče).
  \item Iz vozlišča lahko nadaljujemo le na soseda, ki še ni v maski.
  Tako se nobeno vozlišče na isti poti ne ponovi in dobimo preproste poti.
\end{enumerate}

Ker za vsako vozlišče $v$ sprožimo DFS, algoritem res obišče vse preproste
poti, ki se začnejo v $v$, in vsako pot prešteje natanko enkrat --- tisti,
ki se konča v zadnjem rekurzivnem klicu.
Uporaba bitnih mask nam omogoča, da o obiskanih vozliščih ne vodimo
počasnejših seznamov, pač pa preprosto testiramo posamezne bite.

Časovna zahtevnost je eksponentna v $n$, kar smo potrdili tudi eksperimentalno:
čas izvajanja hitro naraste, ko presežemo približno $n=20$ vozlišč.
Zato je ta funkcija primerna predvsem za:
\begin{itemize}
  \item grafe z največ približno 20--22 vozlišči, kjer jo lahko uporabimo
  tudi za izčrpen pregled;
  \item evalvacijo posameznih kandidatov v metahevristiki,
  kjer število ocen (korakov) omejimo.
\end{itemize}

\subsection{Generiranje vseh kubičnih grafov}

Za manjše $n$ želimo domnevo preveriti izčrpno, zato potrebujemo 
generator vseh neizomorfnih povezanih kubičnih grafov na $n$ vozliščih.

Uporabimo vgrajeni generator v SageMath-u, ki je zgrajen na knjižnici \texttt{nauty}.
S parametri določimo:
\begin{itemize}
  \item da naj ima vsak graf natanko $n$ vozlišč,
  \item da je minimalna in maksimalna stopnja vozlišč enaka $3$,
  \item da je graf povezan,
  \item da nas zanimajo samo neizomorfni grafi.
\end{itemize}

Dobimo generator, ki enega za drugim vrača predstavnike izomorfnih razredov
vseh povezanih 3-regularnih grafov na $n$ vozliščih.
S tem lahko za vsak graf izračunamo $pn(G)$ in določimo globalni minimum.

Eksperimentalno smo izračunali število takih grafov za $n=4,6,\dots,18$,
kar je povzeto v Tabeli~\ref{tab:counts}.

\begin{table}[h]
\centering
\begin{tabular}{cc}
\toprule
$n$ & \# kubičnih grafov \\
\midrule
 4  & 1 \\
 6  & 2 \\
 8  & 5 \\
10  & 19 \\
12  & 85 \\
14  & 509 \\
16  & 4060 \\
18  & 41301 \\
\bottomrule
\end{tabular}
\caption{Število neizomorfnih povezanih kubičnih grafov na $n$ vozliščih.}
\label{tab:counts}
\end{table}

Rast je zelo hitra: že pri $n=18$ imamo več kot $40\,000$ grafov,
pri $n=20$ pa bi jih bilo že več kot pol milijona.
To upravičuje prehod na metahevristične metode za $n \ge 20$.

\subsection{Metahevristika \emph{simulated annealing}}

Za večja $n$ se izčrpen pregled ne izplača,
zato uporabimo metahevristiko \emph{simulated annealing}.
Ta metoda posnema fizični proces ohlajanja snovi:
na začetku dopušča tudi poslabšanja, kasneje pa vedno bolj
preferira izboljšave.

Potrebujemo dve komponenti:

\paragraph{Sosednji graf.}
Iz danega kubičnega grafa $G$ generiramo soseda z eno 2-robno zamenjavo:
izberemo dva roba $\{u,v\}$ in $\{x,y\}$ s štirimi paroma različnimi 
vozlišči in ju preklopimo v eno od alternativnih paritev
$\{u,x\},\{v,y\}$ ali $\{u,y\},\{v,x\}$.
Takšno operacijo imenujemo \emph{double-edge swap}.
Zanima nas samo rezultat, ki:
\begin{itemize}
  \item nima zank ali večrobov,
  \item je še vedno 3-regularen,
  \item je povezan.
\end{itemize}
Če taka zamenjava v omejenem številu poskusov ne obstaja, soseda ne zamenjamo.

\paragraph{Pravilo sprejemanja.}
Začetni graf je običajno $L_n$.
V vsakem koraku izračunamo subpath number trenutnega grafa $G$ (energija $E$)
in kandidata $G'$ (energija $E'$).
Kandidata vedno sprejmemo, če $E' \le E$.
Če je $E' > E$, ga sprejmemo z verjetnostjo
\[
\exp\!\left(-\frac{E'-E}{T}\right),
\]
kjer je $T$ trenutna temperatura.
Temperaturo po vsakem koraku pomnožimo z faktorjem $0 < \alpha < 1$
(na primer $\alpha = 0{,}999$).
S tem je na začetku sprejemanje poslabšanj pogostejše,
proti koncu pa redko, kar omogoča izogib lokalnim minimumom.

Algoritem vrača najboljši graf, ki ga je videl (najmanjši $pn(G)$),
ter potek energij skozi korake, kar smo uporabili za grafično analizo.

\section{Eksperimentalni rezultati}

\subsection{Preverjanje domneve za manjša $n$}

Za manjša $n$ smo uporabili kombinacijo izčrpnega pregleda in usmerjenega iskanja.
Za $n=16$ in $n=18$ smo za vse kubične grafe izračunali $pn(G)$ in shranili grafe,
pri katerih je $pn(G) < pn(L_n)$.
Lastnosti teh rezultatov so povzete v Tabeli~\ref{tab:smalln}.

\begin{table}[h]
\centering
\begin{tabular}{ccccc}
\toprule
$n$ & $pn(L_n)$ & minimalni $pn(G)$ & $pn(\text{star})$ & komentar \\
\midrule
16 & 12744 & 3640 & 3640 & zvezdasti graf doseže globalni minimum \\
18 & 22532 & 7072 & 7072 & zvezdasti graf doseže globalni minimum \\
\bottomrule
\end{tabular}
\caption{Primerjava subpath number za $L_n$, zvezdaste grafe in globalni minimum
pri $n=16$ in $n=18$.}
\label{tab:smalln}
\end{table}

Vidimo, da je pri obeh teh velikostih graf $L_n$ daleč od minimizatorja:
pri $n=16$ ima skoraj štirikrat več poti kot optimalni graf,
pri $n=18$ pa več kot trikrat več.
V obeh primerih globalni minimum doseže ena izmed zvezdastih konstrukcij,
ki ima vizualno izrazito osrednje vozlišče in tri krake s pendant bloki.

To pomeni, da domneva 10 že za $n=16$ in $n=18$ ne drži:
ne samo da $L_n$ ni edini minimizator, sploh ni minimizator.

\subsection{Rezultati metahevristike}

Metahevristiko simulated annealing smo zagnali za
$n = 10,12,14,16,18,20,22$,
pri čemer je bil začetni graf vedno $L_n$.
Število korakov je bilo $1000$,
začetna temperatura $T_0 = 1$,
faktor ohlajanja $\alpha = 0{,}999$.

Najboljše dosežene vrednosti $pn(G)$ (energije) so v Tabeli~\ref{tab:sa}.

\begin{table}[h]
\centering
\begin{tabular}{ccc}
\toprule
$n$ & najboljši $pn(G)$ z SA & čas izvajanja \\
\midrule
10 & 1276  & $\approx 1{,}5$ s \\
12 & 3076  & $\approx 3{,}6$ s \\
14 & 5504  & $\approx 5{,}3$ s \\
16 & 8248  & $\approx 9{,}8$ s \\
18 & 7072  & $\approx 16{,}0$ s \\
20 & 14212 & $\approx 30{,}0$ s \\
22 & 20580 & $\approx 44{,}4$ s \\
\bottomrule
\end{tabular}
\caption{Najboljše vrednosti subpath number, ki jih je našel simulated annealing.}
\label{tab:sa}
\end{table}

Za $n=16$ in $n=18$ lahko te rezultate primerjamo z znanimi globalnimi minimumi:

\begin{itemize}
  \item pri $n=16$ je globalni minimum $3640$, simulated annealing pa je
  v enem zagonu našel graf z $pn(G)=8248$, torej precej boljši od $L_{16}$,
  vendar še ne optimalen;
  \item pri $n=18$ je globalni minimum $7072$, enako pa znaša tudi najboljši
  rezultat simulated annealinga, kar pomeni, da je metahevristika v tem primeru
  dosegla globalni minimum.
\end{itemize}

Potek energije skozi korake (narisan v zvezku) lepo pokaže značilno obnašanje:
na začetku večja nihanja in občasna poslabšanja, proti koncu pa stabilizacija
okoli lokalnega oziroma globalnega minimuma.

\subsection{Primerjava z zvezdastimi družinami}

Za različna soda $n$ smo izračunali $pn(G)$ za zvezdasti družini star1
in star2 in jih primerjali med seboj.
Rezultati za $n = 14, 20, 26, 32, \dots, 98$ so povzeti v Tabeli~\ref{tab:stars}.

\begin{table}[h]
\centering
\begin{tabular}{cccc}
\toprule
$n$ & $pn(\text{star1})$ & $pn(\text{star2})$ & boljši graf \\
\midrule
14 & 7188   & 5504   & star2 \\
20 & 12816  & 11708  & star2 \\
26 & 19596  & 19064  & star2 \\
32 & 27528  & 27572  & star1 \\
38 & 36612  & 37232  & star1 \\
44 & 46848  & 48044  & star1 \\
50 & 58236  & 60008  & star1 \\
56 & 70776  & 73124  & star1 \\
62 & 84468  & 87392  & star1 \\
68 & 99312  & 102812 & star1 \\
74 & 115308 & 119384 & star1 \\
80 & 132456 & 137108 & star1 \\
86 & 150756 & 155984 & star1 \\
92 & 170208 & 176012 & star1 \\
98 & 190812 & 197192 & star1 \\
\bottomrule
\end{tabular}
\caption{Subpath number za dve zvezdasti družini grafov.}
\label{tab:stars}
\end{table}

Vidimo, da je za manjša $n$ (14, 20, 26) boljša konstrukcija star2,
od $n=32$ naprej pa sistematično zmaguje star1.
V obeh primerih se vrednosti $pn(G)$ rastejo približno kvadratno z $n$,
pri čemer je star1 asimptotsko boljši (ima počasnejšo rast).

V kombinaciji z analizo za $n=16$ in $n=18$ lahko sklepamo,
da so zvezdaste konstrukcije zelo močni kandidati za minimizatorje
subpath number v razredu kubičnih grafov pri večjih $n$,
medtem ko grafi $L_n$ že pri zmerno velikih $n$ močno zaostajajo.

\section{Zaključek in nadaljnje delo}

V poročilu smo opisali problem minimizacije subpath number v razredu
kubičnih grafov, družino grafov $L_n$ in prvotno domnevo o njihovi optimalnosti.
Implementirali smo:
\begin{itemize}
  \item konstrukcijo grafov $L_n$ in dveh zvezdastih družin,
  \item natančen, a eksponenten algoritem za izračun $pn(G)$,
  \item generiranje vseh neizomorfnih povezanih kubičnih grafov za majhna $n$,
  \item metahevristični algoritem simulated annealing
  s 2-robno zamenjavo kot sosednjo operacijo.
\end{itemize}

Eksperimentalno smo pokazali:
\begin{itemize}
  \item da se število kubičnih grafov zelo hitro povečuje
  (pri $n=18$ jih je že več kot $40\,000$),
  \item da grafi $L_n$ pri $n=16$ in $n=18$ nikakor niso minimizatorji
        subpath number; zvezdasti grafi imajo tam več kot trikrat manj poti,
  \item da metahevristika simulated annealing uspešno najde grafe z
        bistveno manjšim $pn(G)$ od $pn(L_n)$, v primeru $n=18$
        pa celo globalni minimum,
  \item da zvezdaste konstrukcije star1 in star2 za večja $n$
        sistematično dajejo manjše subpath number in so zato resni kandidati
        za ``prave'' minimalne strukture.
\end{itemize}

Domneva 10 je torej že za relativno majhna $n$ napačna.
Zvezdaste konstrukcije nakazujejo novo mož­no strukturo
minimalnih kubičnih grafov, kjer večina vozlišč leži v pendant blokih,
medtem ko je hrbtenica grafa kratka.

Možne smeri nadaljnjega dela so:
\begin{itemize}
  \item teoretična analiza subpath number za družine $L_n$ in zvezdaste grafe,
        vključno z asimptotskimi ocenami;
  \item izboljšave metahevristike (drugačne sosednje operacije,
        lokalno iskanje, hibridne metode);
  \item poskus dokazati, da je ena izmed zvezdastih družin
        dejansko minimalna v širšem razredu kubičnih grafov,
        vsaj za dovolj velika $n$.
\end{itemize}

\end{document}
