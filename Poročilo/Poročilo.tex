\documentclass[11pt,a4paper]{article}

\usepackage[utf8]{inputenc}
\usepackage[slovene]{babel}
\usepackage{amsmath,amssymb}
\usepackage{graphicx}
\usepackage{geometry}
\usepackage{booktabs}
\usepackage{hyperref}
\geometry{margin=2.5cm}

\title{Subpath number v kubičnih grafih:\\
minimizacija in protimeri za grafe $L_n$}
\author{Žiga Obradović, Lovro Levačić}
\date{\today}

\begin{document}
\maketitle

\begin{enumerate}
      \item uvod s teorijo (subpath in Konstrukcija Ln) (Oba)
      \item Domneva za manjše n do n=18 (število protiprimerov in kakšen protiprimer narisan). Tabela o številu grafov in zakaj sva pognala samo za n = 18 (Žiga)
      \item Opis simulated annealing in katere protiprimere najde (za Ln (SA\_results)) (Žiga)
      \item Domneva za boljše grafe in konstrukcija caterpillar 1 in 2 in drevesa (slike) (Lovro)
      \item Primerjava rezultatov za Caterpillar in tree z Ln (tabela za subpath number). Rezultati SA (ni najdel boljših). Za n=20 je Cat2 najboljši (skok med 18, 20 in 22) (Oba)
      \item Zaključek
\end{enumerate}

\section{Uvod}

V projektu obravnavamo problem minimizacije števila preprostih poti
v razredu kubičnih grafov, torej grafov, kjer ima vsako vozlišče stopnjo $3$.
Naj bo $G=(V,E)$ povezan graf. 
\emph{Subpath number} grafa $G$, označen s $pn(G)$, definiramo kot število vseh
preprostih poti v grafu, pri čemer štejemo tudi trivialne poti
(zgolj eno vozlišče).

Preprosta pot je zaporedje različnih vozlišč
\[
(v_0, v_1, \dots, v_\ell),
\]
kjer sta vsaki zaporedni vozlišči povezni z robom grafa.
Ker je graf neorientiran, poti, ki gredo v nasprotni smeri, štejemo ločeno.

V literaturi je bila postavljena domneva, da za vsako sodo število vozlišč
$n \ge 10$ obstaja natanko en kubični graf $L_n$, ki minimizira subpath number
med vsemi kubičnimi grafi na $n$ vozliščih.
Znano je, da domneva za dovolj velika $n$ ne drži, ni pa znano,
pri katerem najmanjšem $n$ se pojavi prva protimera.

V projektu smo želeli:
\begin{itemize}
  \item za manjša $n$ domnevo preveriti izčrpno,
  \item za večja $n$ uporabiti metahevristiko \emph{simulated annealing}
        in poiskati grafe z manjšim $pn(G)$ od $pn(L_n)$,
  \item konstruirati naravne družine kubičnih grafov in jih
        primerjati z $L_n$.
\end{itemize}

\section{Teoretično ozadje}

\subsection{Subpath number}

Za poljuben povezan graf $G=(V,E)$ definiramo
\[
pn(G) = \#\{\text{preprostih poti v } G\},
\]
pri čemer kot poti štejemo tudi trivialne poti dolžine $0$.

Število poti zelo hitro raste.
Za ilustracijo: en sam rob ima $pn(G)=4$, trikotnik $pn(G)=15$,
kvadrat $pn(G)=28$, popoln graf $K_4$ pa $pn(G)=64$.
To nakazuje, da je izračun $pn(G)$ v splošnem eksponenten v številu vozlišč,
zato je natančen izračun smiseln predvsem za grafe z zmernim $n$
(nekje do približno 20 vozlišč).

\subsection{Družina grafov $L_n$ in domneva}

Grafi $L_n$ so definirani za soda $n \ge 10$ in so zgrajeni iz več kopij grafa
$K_4 - e$ (popoln graf na štirih vozliščih, iz katerega odstranimo en rob),
ki so povezane v linearno verigo. Na koncih verige so dodani t.\ i.\ pendant bloki.

Natančneje:
\begin{itemize}
  \item če je $n = 4q - 2$, je $L_n$ sestavljen iz $(n-10)/4$ kopij grafa $K_4 - e$, na obeh koncih pa sta pendant bloka na $5$ vozlišč;
  \item če je $n = 4q$, je $L_n$ sestavljen iz $(n-12)/4$ kopij grafa $K_4 - e$, na koncih pa sta pendant bloka na $5$ in $7$ vozlišč.
\end{itemize}

Geometrijsko gledano dobimo verigo blokov $K_4-e$, ki se na obeh koncih
zaključi z nekoliko večjima komponentama (Slika~\ref{fig:Ln}).

Domneva, ki jo preverjamo, je:

\medskip
\noindent
\textbf{Domneva 10.}
\emph{V razredu kubičnih grafov na $n$ vozliščih, kjer je $n$ sodo,
je graf $L_n$ edini graf, ki minimizira subpath number.}
\medskip

Ker je bilo že dokazano, da domneva za dovolj velika $n$ odpove,
nas zanima predvsem, pri katerih najmanjših $n$ se to zgodi.

\section{Metodologija}

\subsection{Konstrukcija družin grafov}

Osnovni gradnik v vseh naših konstrukcijah je graf $K_4 - e$,
tj.\ popolni graf na štirih vozliščih, iz katerega odstranimo en rob.
V takem grafu imata dve vozlišči stopnjo $2$, preostali dve pa stopnjo $3$.
Ta dva vozlišča stopnje $2$ uporabljamo kot “priklopni točki”,
prek katerih gradnike lepimo skupaj.

V kodi (\texttt{funkcije2.py}) smo gradnike poimenovali
\emph{levi gradnik}, \emph{srednji gradnik} in \emph{desni gradnik}
(ter njihove variante z dvema priklopnima vozliščema).
Ideja je vedno ista: na $K_4-e$ po potrebi dodamo nekaj novih vozlišč,
da dobimo pendant blok na $5$ ali $7$ vozliščih, ki ima eno ali dve
priklopni vozlišči.
Pomožni funkciji \texttt{add\_gadget} in \texttt{add\_gadget2}
poskrbita za pravilno oštevilčenje in povezavo gradnikov.

Na tej osnovi definiramo tri družine kubičnih grafov.

\paragraph{Družina $L_n$.}
Graf $L_n$ zgradimo v skladu z definicijo:
začnemo z levim gradnikom, dodajamo verigo srednjih gradnikov
in verigo zaključimo z levim ali desnim gradnikom tako, da dobimo
kubični graf na $n$ vozliščih.
Za različne ostanke $n$ modulo $4$ dobimo različne dolžine verige.
Primeri grafov $L_n$ za $n = 10,12,\dots,24$ so na Sliki~\ref{fig:Ln}.

\begin{figure}[h]
\centering
\includegraphics[width=0.9\textwidth]{slike/Ln_grafi.png}
\caption{Primeri grafov $L_n$ za soda $n$ med $10$ in $24$.}
\label{fig:Ln}
\end{figure}

\paragraph{Goseničasta (zvezdasta) družina.}
Druga konstrukcija (\texttt{build\_caterpillar}) uporablja iste gradnike,
vendar jih ne zložimo v eno verigo, temveč dobimo bolj zvezdasto strukturo.
Odvisno od $n \bmod 6$ začnemo z levim ali desnim gradnikom,
po potrebi vstavimo še nekaj vmesnih gradnikov, na koncu pa ponovno zaključimo
z levim ali desnim gradnikom.
Tipičen graf ima nekaj “krakov” z pendant bloki in bolj gosto jedro
(Slika~\ref{fig:star}).

\begin{figure}[h]
\centering
\includegraphics[width=0.9\textwidth]{slike/star_grafi.png}
\caption{Goseničasti (zvezdasti) grafi za soda $n$ med $18$ in $46$
(konstrukcija \texttt{build\_caterpillar}).}
\label{fig:star}
\end{figure}

\paragraph{Drevesna družina $\mathrm{Tree}_n$.}
Tretja družina izhaja iz dejanskega drevesa.
Najprej zgradimo drevo $T_k$ na $k$ vozliščih:
v korenu imamo vozlišče stopnje $3$, nato pa postopoma razvejamo liste,
dokler ne dosežemo želenega števila vozlišč.
Na koncu ima drevo veliko listov stopnje $1$.

Funkcija \texttt{build\_tree} potem na vsak list pripne gradnik tipa $K_4-e$
(z dvema priklopnima vozliščema) in ga z obema priklopnima vozliščema
poveže na isti list. S tem dvignemo stopnjo lista iz $1$ na $3$,
novi vrhovi gradnikov pa so prav tako stopnje $3$.
Tako dobimo kubični graf na $n$ vozliščih, ki ga označimo z
$\mathrm{Tree}_n$ (Slika~\ref{fig:tree-layout}).

\begin{figure}[h]
\centering
\includegraphics[width=\textwidth]{slike/tree_layout.png}
\caption{Primer grafa iz družine $\mathrm{Tree}_n$ v drevesnem razporedu.}
\label{fig:tree-layout}
\end{figure}


\subsection{Izračun subpath number}

Za izračun $pn(G)$ smo napisali funkcijo \texttt{subpath\_number(G)}.
Ideja je naslednja:
\begin{itemize}
  \item vozlišča oštevilčimo z $0,\dots,n-1$ in množico obiskanih vozlišč
        predstavimo z bitno masko;
  \item zgradimo seznam sosedov;
  \item za vsako vozlišče $v$ zaženemo globinsko iskanje (DFS),
        ki po grafu sprehaja vse preproste poti, ki se začnejo v $v$;
  \item pri vsakem rekurzivnem klicu pot, ki se konča v trenutnem vozlišču,
        štejemo kot eno (tako dobimo tudi trivialne poti),
        nadaljujemo pa samo na sosede, ki še niso v maski.
\end{itemize}

Tako vsako pot preštejemo natanko enkrat.
Ker moramo v najslabšem primeru obravnavati ogromno število poti,
je časovna zahtevnost eksponentna.
V praksi to pomeni, da je izračun še izvedljiv za grafe s približno 20
(oz.\ kvečjemu 22) vozlišči; za večje grafe ga uporabljamo le v kombinaciji
z metahevristiko in omejenim številom korakov.

\subsection{Generiranje vseh kubičnih grafov}

Za manjša $n$ želimo domnevo preveriti izčrpno, zato potrebujemo
generator vseh neizomorfnih povezanih kubičnih grafov na $n$ vozliščih.
Uporabimo vgrajeno funkcijo \texttt{graphs.nauty\_geng}
iz SageMath-a (program \emph{geng} iz paketa \emph{nauty}).

V funkciji \texttt{cubic\_graphs(n)} nastavimo parametre tako, da:
\begin{itemize}
  \item ima vsak graf natanko $n$ vozlišč,
  \item je minimalna in maksimalna stopnja vozlišč enaka $3$,
  \item je graf povezan,
  \item dobimo po en predstavnik iz vsakega izomorfnega razreda.
\end{itemize}

Za soda $n$ od $4$ do $22$ smo prešteli, koliko takih grafov dobimo.
Rezultati so v Tabeli~\ref{tab:counts}.

\begin{table}[h]
\centering
\begin{tabular}{cc}
\toprule
$n$ & \# kubičnih grafov \\
\midrule
 4  & 1 \\
 6  & 2 \\
 8  & 5 \\
10  & 19 \\
12  & 85 \\
14  & 509 \\
16  & 4060 \\
18  & 41301 \\
20  & 510489 \\
22  & 7319447 \\
\bottomrule
\end{tabular}
\caption{Število neizomorfnih povezanih kubičnih grafov na $n$ vozliščih.}
\label{tab:counts}
\end{table}

Rast je zelo hitra: pri $n=18$ imamo že več kot $40\,000$ grafov,
pri $n=20$ več kot pol milijona, pri $n=22$ pa več milijonov.
To je glavni razlog, da smo za $n \ge 20$ prešli na metahevristične metode.

\subsection{Metahevristika simulated annealing}

Za večja $n$ domneve ne moremo preverjati izčrpno,
zato smo uporabili \emph{simulated annealing} (SA).
Gre za naključno iskanje, pri katerem na začetku sprejemamo
tudi poslabšanja, kasneje pa skoraj samo še izboljšave.

\paragraph{Sosednji graf.}
Na prostoru povezanih kubičnih grafov definiramo sosesko
z eno 2-robno zamenjavo (double-edge swap), implementirano
v funkciji \texttt{random\_cubic\_neighbor}.
Iz trenutnega grafa $G$ naključno izberemo dva roba
$\{u,v\}$ in $\{x,y\}$ s štirimi različnimi vozlišči,
oba roba odstranimo in ju nadomestimo z eno od dveh
novih paritev ($\{u,x\},\{v,y\}$ ali $\{u,y\},\{v,x\}$).
Kandidat sprejmemo le, če ne nastanejo zanke ali večrobi
in graf ostane povezan ter 3-regularen.

\paragraph{Energija in temperatura.}
Energija stanja je $E(G) = pn(G)$.
Če ima sosednji graf $G'$ manjšo ali enako energijo, ga vedno sprejmemo.
Če je $E' > E$, ga sprejmemo z verjetnostjo
\[
\exp\!\left(-\frac{E'-E}{T}\right),
\]
kjer je $T$ trenutna temperatura.
Temperaturo znižujemo po eksponentni shemi
$T_{k+1} = \alpha T_k$ z $\alpha < 1$.
Parameter $T_0$ izberemo tako, da na začetku
razmeroma pogosto sprejemamo poslabšanja, ob koncu pa skoraj nikoli.

SA smo v praksi poganjali z nekaj deset tisoč koraki,
kot začetni graf pa smo vzeli bodisi $L_n$ bodisi kak graf
iz družine $\mathrm{Tree}_n$.

\section{Eksperimentalni rezultati}

\subsection{Preverjanje domneve za manjša $n$}

Za soda $n$ do $18$ smo s funkcijo \texttt{cubic\_graphs(n)}
pridobili vse neizomorfne povezane kubične grafe na $n$ vozliščih
in za vsak graf $G$ izračunali $pn(G)$.
Za vsak $n$ smo posebej primerjali $pn(G)$ z $pn(L_n)$.

Za $n=16$ in $n=18$ smo izračun izvedli v celoti.
Pri $n=16$ smo obdelali vseh 4060 grafov,
pri $n=18$ pa več kot 40 tisoč grafov.
V obeh primerih smo našli več kubičnih grafov
z manjšim subpath number kot pri $L_n$.
To pomeni, da $L_{16}$ in $L_{18}$ nista minimizatorja,
zato domneva v tej obliki že pri teh vrednostih odpove.

\subsection{Simulated annealing za večja $n$}

Za $n \ge 20$ izčrpen pregled ni več smiseln, zato smo uporabili SA.
Algoritem smo zagnali za več sodih $n$ med $20$ in $32$,
in sicer z različnimi začetnimi grafi
(grafi $L_n$ in grafi $\mathrm{Tree}_n$).

Pri nekaterih manjših $n$ (npr.\ $n=10,12,14$) se je pokazalo,
da SA iz $L_n$ ne najde boljšega grafa; energija se
v daljših tekih praktično ne spremeni.
To je skladno z intuicijo, da so grafi $L_n$ pri majhnih $n$
vsaj lokalno dobri.

Za večja $n$ smo dobili bolj razgibano sliko.
Pri nekaterih parametrih je SA precej znižal energijo
v prvih fazah in našel grafe z manjšim $pn(G)$ od izbranega začetnega.
Kot zanimiv primer izpostavimo $n=32$, kjer smo začeli v grafu
$\mathrm{Tree}_{32}$.
V daljšem teku (okoli 20 tisoč korakov) se energija ni znižala,
kar kaže, da je $\mathrm{Tree}_{32}$ v izbrani soseski
zelo dober lokalni minimum.
Podobno obnašanje smo opazili tudi pri nekaterih drugih
drevesnih konstrukcijah.

Zaradi časovne zahtevnosti izračuna $pn(G)$ pri večjih $n$
rezultatov SA za zdaj razumemo predvsem kot indikativne:
dajo nam kandidate za dobre grafe, ne pa nujno globalnih minimizatorjev.

\section{Zaključek}

V poročilu smo opisali, kako smo pristopili k problemu minimizacije
subpath number v razredu kubičnih grafov.
Najprej smo implementirali natančen izračun $pn(G)$
in generator vseh kubičnih grafov na danem številu vozlišč.
Na tej osnovi smo:

\begin{itemize}
  \item prešteli vse neizomorfne povezane kubične grafe za soda $n$ do $22$,
  \item za $n=16$ in $n=18$ izčrpno primerjali $L_n$ z vsemi kubičnimi grafi
        in našli več grafov z manjšim subpath number, 
  \item konstruirali dve dodatni družini kubičnih grafov
        (goseničasto družino in drevesno družino $\mathrm{Tree}_n$),
  \item za večja $n$ uporabili simulated annealing in preverili,
        ali lahko iz danega začetnega grafa pridemo do grafa
        z manjšim $pn(G)$.
\end{itemize}

Glavno sporočilo je, da grafi $L_n$ že pri razmeroma majhnih $n$
niso minimizatorji subpath number med vsemi kubičnimi grafi.
Za večja $n$ se kombinacija konstrukcijskih družin in SA
izkaže za uporaben pristop: omogoča nam, da vsaj približno
raziskujemo prostor kubičnih grafov, čeprav natančen izračun
$pn(G)$ ostaja ozko grlo.

V nadaljevanju bi bilo zanimivo poskusiti še z drugimi metahevristikami
(npr.\ tabu iskanjem ali genetskimi algoritmi) in hkrati poiskati
bolj pametne aproksimacije za $pn(G)$,
da bi lahko obravnavali tudi grafe z bistveno več vozlišči.

\end{document}