documentclass[11pt,a4paper]{article}

\usepackage[utf8]{inputenc}
\usepackage[slovene]{babel}
\usepackage{amsmath,amssymb}
\usepackage{graphicx}
\usepackage{geometry}
\usepackage{booktabs}
\usepackage{hyperref}
\usepackage{float}
\usepackage[table]{xcolor}
\usepackage{siunitx}
\sisetup{
  group-separator = {.},
  group-minimum-digits = 3
}
\geometry{margin=2.5cm}

\title{Tukaj pišem svoj del, da bova kasneje združila}
\author{Lovro}
\date{\today}

\begin{document}
\maketitle

\section{Uvod}

Naj bo $G=(V,E)$ povezan, neorientiran graf. 
\emph{Subpath number} grafa $G$, označen s $pn(G)$, definiramo kot število
vseh preprostih poti v grafu, pri čemer štejemo tudi trivialne poti
(zgolj eno vozlišče).
Preprosta pot je zaporedje različnih vozlišč
\[
(v_0, v_1, \dots, v_\ell),
\]
kjer sta vsaki zaporedni vozlišči povezani z robom grafa; poti, ki gredo
v nasprotni smeri, štejemo ločeno.
Za en sam rob dobimo $pn(G)=4$, za trikotnik $pn(G)=15$,
za popoln graf $K_4$ pa že $pn(G)=64$,
kar nakazuje, da $pn(G)$ v splošnem raste eksponentno s številom vozlišč.

V tem projektu nas zanimajo \emph{kubični grafi},
tj.\ povezani grafi, v katerih ima vsako vozlišče stopnjo $3$.
Za sodo število vozlišč $n$ želimo med vsemi kubičnimi grafi na $n$ vozliščih
najti tiste, ki minimizirajo $pn(G)$.

V literaturi je bila predlagana posebna družina kubičnih grafov $L_n$.
Grafi $L_n$ so definirani za soda $n \ge 10$ in so zgrajeni iz verige
blokov $K_4-e$, kjer $K_4-e$ pomeni popoln graf na štirih vozliščih,
iz katerega odstranimo en rob.
Verigi dodamo ustrezna pendant bloka na obeh koncih,
tako da dobimo povezano 3-regularno strukturo na natančno $n$ vozliščih.

\medskip

Osrednja izhodiščna trditev je bila:

\medskip
\noindent
\textbf{Domneva 10.}
\emph{Naj bo $n \ge 10$ sodo. Med vsemi povezanimi kubičnimi grafi
na $n$ vozliščih je graf $L_n$ edini graf, ki minimizira subpath number.}

\medskip
Znano je, da domneva za dovolj velika $n$ ne drži, ne ve se pa,
pri katerih najmanjših $n$ se pojavi prvi protiprimer
in kakšna je struktura grafov z majhnim $pn(G)$.

\section{Domneva za boljše grafe in nove konstrukcije}

Rezultati iz prejšnjih razdelkov kažejo, da grafi $L_n$ že pri $n=16$
in $n=18$ niso minimizatorji subpath number,
za večja $n$ pa imajo naše konstrukcije (Cat1, Cat2, $\mathrm{Tree}_n$)
veliko manjše število podpoti od $L_n$.
V tem razdelku na kratko opišemo, kako so te družine zgrajene,
in formuliramo novo domnevo o njihovem obnašanju.

\subsection{Osnovni gradniki na osnovi $K_4-e$}

Skupni temelj vseh konstrukcij je graf $K_4-e$:
popoln graf na štirih vozliščih, iz katerega odstranimo en rob.
V njem imata dve vozlišči stopnjo $2$, preostali dve pa stopnjo $3$.
Ta vozlišča stopnje $2$ interpretiramo kot \emph{priklopni vozlišči},
prek katerih gradnik lepimo na preostanek grafa.

Na $K_4-e$ dodajamo manjša drevesa, s čimer dobimo tri tipe blokov:
\begin{itemize}
  \item \emph{pendant blok} na $5$ vozliščih, ki se na graf priklopi v eni točki,
  \item \emph{podaljšani blok} na $7$ vozliščih, kjer pendantu dodamo še
        par vmesnih vozlišč,
  \item \emph{srednji blok} z dvema priklopnima vozliščema,
        ki služi kot člen v verigi oziroma hrbtenici grafa.
\end{itemize}
Pri vseh konstrukcijah skrbimo, da imajo končni grafi stopnjo $3$
v vsakem vozlišču in so povezani.

\subsection{Goseničasti konstrukciji Cat1 in Cat2}

Goseničasti družini Cat1 in Cat2 sta zgrajeni tako,
da večino vozlišč dobimo v pendant blokih,
povezanih na razmeroma kratko hrbtenico.

V različici \textbf{Cat1} postavimo v središče vozlišče stopnje $3$
in iz njega speljemo več krakov.
Na konce krakov pripnemo pendant bloke, vmes pa po potrebi
dodajamo srednje in podaljšane bloke.
Dobljeni grafi imajo izrazito zvezdasto obliko:
majhno jedro in več daljših krakov

V različici \textbf{Cat2} uporabljamo iste gradnike,
vendar drugače razporedimo srednje in podaljšane bloke,
zlasti pri velikostih $n = 6k+2$.
Tam konstrukcija Cat2 zamenja nekaj pendant blokov
z drugačno kombinacijo srednjih in podaljšanih blokov
in s tem spremeni število podpoti.
Razlika med Cat1 in Cat2 je lepo vidna pri $n=26$
-(To vidm da si ze neki omenju, mogoce samo zdruziva)-

Če primerjamo $\mathrm{pn}(\text{Cat1})$ in $\mathrm{pn}(\text{Cat2})$,
vidimo tipičen vzorec:
za manjša $n$ in posebej za $n = 6k+2$ je pogosto boljša Cat2,
za večja $n$ pa sistematično prevlada Cat1.
V obeh družinah subpath number raste približno kvadratno z $n$,
a Cat1 ima pri velikih $n$ počasnejšo rast.

\subsection{Drevesna družina $\mathrm{Tree}_n$}

Družina $\mathrm{Tree}_n$ izhaja iz povsem drugačne,
drevesne strukture.
Najprej zgradimo drevo $T_k$:
v korenu je vozlišče stopnje $3$,
nato pa zaporedno razvejamo liste, dokler ne dosežemo želenega $k$.
Tako dobimo razmeroma plitvo drevo z velikim številom listov.

V naslednjem koraku na vsak list pripnemo gradnik na osnovi $K_4-e$
z dvema priklopnima vozliščema.
Oba priklopna vrha povežemo na isti list,
s čimer dvignemo njegovo stopnjo iz $1$ na $3$,
novi vrhovi gradnika pa so prav tako stopnje $3$.
Končni graf je kubičen in ga označimo z $\mathrm{Tree}_n$.

Struktura $\mathrm{Tree}_n$ je zelo različna od Cat1/Cat2:
namesto nekaj krakov imamo cel “gozd” pendant blokov na listih drevesa,
zgornji del drevesa pa ostane kratek.
Kljub razlikam numerični rezultati kažejo,
da imata $\mathrm{Tree}_n$ in Cat1 za vrsto $n$
enako vrednost subpath number,
kar nakazuje globljo povezavo med konstrukcijama.

\subsection{Nova domneva}

\begin{itemize}
  \item za $n \in \{10,12,14,16,18,20\}$ je graf Cat2
        (in z njim tudi Cat1/Tree, kjer imajo enak $pn(G)$)
        boljši od $L_n$, pri čemer sta za $n \le 20$
        zares pregledana \emph{vsa} kubična grafa,
  \item za večja $n$ (do vrednosti, kjer smo računali) imajo Cat1,
        Cat2 in $\mathrm{Tree}_n$ za dano $n$ vedno
        veliko manjše število podpoti kot $L_n$,
  \item numerično se subpath number Cat1 in $\mathrm{Tree}_n$
        ujema na dolgem intervalu $n$, medtem ko Cat2 izstopa
        predvsem pri $n = 6k+2$.
\end{itemize}

Na podlagi teh opazovanj lahko neformalno zapišemo novo domnevo:

\medskip
\noindent
\textbf{Domneva.}
\emph{Za majhna soda $n$ kubični graf z najmanjšim subpath number
pripada eni izmed družin Cat1, Cat2 ali $\mathrm{Tree}_n$.
Za $n \le 20$ je tak graf enak grafu Cat2,
za večja $n$ pa družini Cat1 in $\mathrm{Tree}_n$
dasta kandidata, ki ju z našimi metodami ni uspelo izboljšati.}

\medskip
Ta domneva povzema numerične rezultate in nadomesti prvotno,
preveč optimistično domnevo o optimalnosti grafov $L_n$.

\section{Zaključek}

V projektu smo preučevali minimizacijo subpath number v razredu
povezanih kubičnih grafov.
Za $n \in \{10,12,14,16,18\}$ smo s pomočjo programa \texttt{nauty\_geng}
pregledali vse kubične grafe in izračunali njihov $pn(G)$.
Pokazali smo, da grafi $L_n$ za $n=10,12,14$ res dosežejo minimum,
za $n=16$ in $n=18$ pa smo našli več protiprimerov,
tako da izvorna domneva ne drži.

Za večja $n$ smo uporabili metahevristiko \emph{simulated annealing},
ki iz različnih začetnih grafov (tudi iz $L_n$)
najde grafe z bistveno manjšim subpath number.
Na tej podlagi smo definirali tri nove konstrukcijske družine:
goseničasti konstrukciji Cat1 in Cat2 ter drevesno družino $\mathrm{Tree}_n$,
vse zgrajene iz gradnikov na osnovi $K_4-e$.

Primerjava subpath number v tabelah pokaže,
da imajo te konstrukcije za isto $n$ velikostno razredno manjše
število podpoti kot grafi $L_n$.
Za $n \le 20$ je Cat2 dokazano optimalen,
za večja $n$ pa Cat1 in $\mathrm{Tree}_n$ predstavljata zelo močne kandidate,
ki jih z dosedanjimi metodami nismo uspeli izboljšati.

Skupni zaključek je, da grafi $L_n$ niso pravi minimizatorji subpath number
in da strukture z veliko pendant bloki (Cat1, Cat2, $\mathrm{Tree}_n$)
veliko bolje izkoriščajo omejitve kubičnih grafov pri zmanjševanju
števila preprostih poti.

\end{document}

