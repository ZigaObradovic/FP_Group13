\documentclass[11pt,a4paper]{article}

\usepackage[utf8]{inputenc}
\usepackage[slovene]{babel}
\usepackage{amsmath,amssymb}
\usepackage{graphicx}
\usepackage{geometry}
\usepackage{booktabs}
\usepackage{hyperref}
\usepackage{float}
\geometry{margin=2.5cm}

\title{Tukaj pišem svoj del, da bova kasneje združila}
\author{Žiga}
\date{\today}

\begin{document}
\maketitle

\section{Preverjanje domneve za majhne n}

Če smo domnevo želeli preveriti na vseh kubičnih grafih na $n$ vozliščih, smo morali najprej te grafe generirati. Izkazalo se je, da številov kubičnih grafov na $n$ vozliščih eksponentno raste s številom vozlišč, zato smo domnevo preverili na vseh kubičnih grafih le za $n \leq 18$.

\begin{table}[H]
\centering
\begin{tabular}{cc}
\toprule
$n$ & \# kubičnih grafov \\
\midrule
10  & 19 \\
12  & 85 \\
14  & 509 \\
16  & 4060 \\
18  & 41301 \\
20  & 510489 \\
22  & 7319447 \\
\bottomrule
\end{tabular}
\caption{Število neizomorfnih povezanih kubičnih grafov na $n$ vozliščih.}
\label{tab:st_grafov}
\end{table}

Za $n \in \{10, 12, 14, 16, 18\}$ smo pridobili vse neizomorfne povezane kubične grafe na $n$ vozliščih in za vsak graf $G$ izračunali $pn(G)$. Za vsak $n$ smo posebej primerjali število podpoti generiranega grafa $pn(G)$ s številom podpoti grafa $L_n$ $pn(L_n)$. Rezultati so prikazani v spodnji tabeli.

\begin{table}[H]
\centering
\begin{tabular}{cccc}
\toprule
$n$ & število boljših grafov & minimalno število podpoti & $pn(L_n)$\\
\midrule
10  & 0 & 1276 & 1276\\
12  & 0 & 3076 & 3076\\
14  & 0 & 5504 & 5504\\
16  & 6 & 3640 & 12744\\
18  & 23 & 7072 & 22532\\
\bottomrule
\end{tabular}
\caption{Število vseh grafov na $n$ vozliščih z manjšim številom podpoti od grafa $L_n$.}
\label{tab:rezultat_generiranja}
\end{table}

Na podlagi rezultatov vidimo, da domneva velja za $n \in \{10, 12, 14\}$, medtem ko lahko domnevo za $n \in \{16, 18\}$ zavrnemo. Namreč v primeru ko je $n=16$ smo našli 6 grafov z manjišim številom podpoti od grafa $L_{16}$, v primeru $n = 18$ pa že kar 23 grafov z manjišim številom podpoti od grafa $L_{18}$.

\begin{figure}[H]
\centering
\includegraphics[width=0.85\textwidth]{slike/n18_grafi.png}
\caption{Grafi z najmanjšim številom podpoti za $n=18$.}
\label{fig:n18_grafi}
\end{figure}

Ker je za $n \geq 20$ izračun \emph{subpath number} vedno zahtevnejši, grafov pa vedno več, bomo za iskanje protiprimerov uporabili drugačno metodo. Ta metoda se mimenuje \emph{simulated annealing}. 

\section{Simulated annealing}

Za iskanje kubičnih grafov z manjšim številom podpoti od $L_n$ grafa smo za $n \geq 20$ uporabili metahevristični algoritem \emph{simulated annealing} (SA). Gre za 
probabilistično metodo globalne optimizacije, ki posnema proces 
fizikalnega ohlajanja kovin: sistem se sprva nahaja pri visoki temperaturi 
in lahko sprejema tudi poslabšanja, postopoma pa se temperatura znižuje, 
kar zmanjšuje verjetnost sprejemanja slabših rešitev. Pri dovolj počasnem 
ohlajanju se algoritem z visoko verjetnostjo približa globalnemu minimumu
energetske funkcije, katera je v našem primeru \emph{subpath number}.

\subsection{Delovanje}

V našem primeru je prostor iskanja sestavljen iz vseh \emph{povezanih 
kubičnih grafov} na $n$ vozliščih. Prehod med grafi definiramo s t.\ i. 
\emph{dvojnim prevezovanjem robov} (ang.~double-edge swap). Naj bo 
$G = (V,E)$ kubičen graf in naj bosta izbrani dve disjunktni povezavi 
$\{u,v\},\{x,y\} \in E$. Zamenjava poteka tako, da se povezavi odstranita 
in nadomestita z novima paroma $\{u,x\}$ in $\{v,y\}$ ali s paroma 
$\{u,y\}$ in $\{v,x\}$. S takimi menjavami povezav ohranimo 3-regularnost grafa. Energijska funkcija, ki jo minimiziramo, pa je v našem primeru podana z $E(G) = \text{subpath\_number}(G),$
kjer funkcija $\text{subpath\_number}(\cdot)$ prešteje vse različne 
podpoti v grafu.\\

Proces začnemo z grafom, ki ga želimo izboljšati. V našem primeru je to graf $L_n =: G$. Pri vsakem koraku z double edge swap generiramo novega soseda $G'$ in izračunamo $\Delta E = E(G') - E(G)$.
Če je $\Delta E \le 0$, rešitev sprejmemo. V nasprotnem primeru jo sprejmemo z verjetnostjo $$p = \exp\!\left(-\frac{\Delta E}{T}\right),$$
kjer $T>0$ predstavlja trenutni temperaturni parameter. 
S tem omogočimo kontrolirano sprejemanje slabših rešitev zgodaj v 
postopku, kar preprečuje prezgodnjo ujetost v lokalne minimume. Če rešitev sprejmemo nastavimo $G := G'$ in postopek ponavljamo.\\

V našem primeru smo naredili 20000 ponovitev postopka, začetni parameter $T_0$ pa je bil nastavljen tako, da se je v zgodnjih ponovitvah slabša rešitev sprejela z verjetnostjo približno 40 \%, proti koncu postopka pa skoraj nikoli.

\subsection{Rezultati}

Z metodo \emph{simulated annealing} smo za $n \in \{20, 22, \dots, 30\}$ iskali grafe z manjšim številom podpoti od grafa $L_n$.

\begin{table}[H]
\centering
\begin{tabular}{cccc}
\hline
n & $subpath\_number(L_n)$ & boljši $subpath\_number$ & razlika \\
\hline
20 & 51532 & 11708 & 38716 \\
22 & 90760 & 7156 & 83604 \\
24 & 206800 & 12220 & 194580 \\
26 & 363788 & 21760 & 342028 \\
28 & 827988 & 29568 & 798420 \\
30 & 1456016 & 18520 & 1437496 \\
\hline
\end{tabular}
\caption{Rezultati SA.}
\label{tab:Results_SA}
\end{table}

\begin{figure}[H]
\centering
\includegraphics[width=0.85\textwidth]{slike/SA_potek.png}
\caption{Potek energije med simulated annealing za n = 20.}
\label{fig:SA_potek}
\end{figure}

Na sliki lahko vidimo potek energije (števila podpoti) med procesom SA za $n=20$. Kar lahko razberemo je, da so se med postopkom sprejele tudi slabše rešitve, saj energija (število podpoti) ni strogo padajoča. Program je uspešno sprejemal tudi sabše rešitve in se s tem izognil morebitnim lokalnim minimumom. Vselej pa smo na koncu našli graf s kar 5-krat manjšim številom podpoti ob grafa $L_{20}$. Čeprav samo na podlagi rezultatov SA še ne moremo trditi, da je 11706 najmanjše število podpoti za $n = 20$, bomo kasneje dokazali, da je to pravzaprav res in da smo s postopkom SA našli graf, ki minimizira število podpoti za $n=20$.

\begin{figure}[H]
\centering
\includegraphics[width=0.4\textwidth]{slike/cat2_20.png}
\caption{Graf dobljen s SA za $n = 20$.}
\label{fig:Cat2_20}
\end{figure}

Kot pričakovano za $n \in \{10, 12, 14\}$ protiprimerov nismo našli, medtem ko smo za vse večje $n$ našli grafe s precej nižjim številom podpoti. Rezultati so prikazani v zgornji tabeli.\\

Vredno je poudariti, da v splošnem z najdenimi grafi še zdaleč ne minimiziramo števila podpoti za inzbrani $n$, čeprav so za nekatere $n$ najdeni grafi že precej dobri. Našli smo le primere grafov z manjšim številom podpoti od opazovanega $L_n$ grafa. To se dobro vidi, ko primerjamo dobljeno minimalno število podpoti za $n=28$ in $n=30$. Dobljeni graf za $n=30$ ima namreč kar za tretjino manj podpoti od dobljenega grafa za $n=28$. Ta podatek nam za graf z $n=30$ sicer ne pove veliko, medtem ko smo lahko skoraj prepričani, da za $n=28$ obstaja graf z manjšim številom podpoti.



\begin{figure}[H]
\centering
\includegraphics[width=0.85\textwidth]{slike/SA_grafi.png}
\caption{Grafi pridobljeni s SA za $n =$ 26, 28 in $n=18$.}
\label{fig:SA_grafi}
\end{figure}


\end{document}