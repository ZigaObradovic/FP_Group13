\documentclass[11pt,a4paper]{article}

\usepackage[utf8]{inputenc}
\usepackage[slovene]{babel}
\usepackage{amsmath,amssymb}
\usepackage{graphicx}
\usepackage{geometry}
\usepackage{booktabs}
\usepackage{hyperref}
\geometry{margin=2.5cm}

\title{Tukaj pišem svoj del, da bova kasneje združila}
\date{\today}

\begin{document}
\maketitle

\section{Simulated annealing}

Za iskanje kubičnih grafov z manjšim številom podpoti od $L_n$ grafa smo uporabili metahevristični algoritem \emph{simulated annealing} (SA). Gre za 
probabilistično metodo globalne optimizacije, ki posnema proces 
fizikalnega ohlajanja kovin: sistem se sprva nahaja pri visoki temperaturi 
in lahko sprejema tudi poslabšanja, postopoma pa se temperatura znižuje, 
kar zmanjšuje verjetnost sprejemanja slabših rešitev. Pri dovolj počasnem 
ohlajanju se algoritem z visoko verjetnostjo približa globalnemu minimumu
energetske funkcije, katera je v našem primeru \emph{subpath number}.

\subsection{Delovanje}

V našem primeru je prostor iskanja sestavljen iz vseh \emph{povezanih 
kubičnih grafov} na $n$ vozliščih. Prehod med grafi definiramo s t.\ i. 
\emph{dvojnim prevezovanjem robov} (ang.~double-edge swap). Naj bo 
$G = (V,E)$ kubičen graf in naj bosta izbrani dve disjunktni povezavi 
$\{u,v\},\{x,y\} \in E$. Zamenjava poteka tako, da se povezavi odstranita 
in nadomestita z novima paroma $\{u,x\}$ in $\{v,y\}$ ali s paroma 
$\{u,y\}$ in $\{v,x\}$. S takimi menjavami povezav ohranimo 3-regularnost grafa. Energijska funkcija, ki jo minimiziramo, pa je v našem primeru podana z $E(G) = \text{subpath\_number}(G),$
kjer funkcija $\text{subpath\_number}(\cdot)$ prešteje vse različne 
podpoti v grafu.\\

Proces začnemo z grafom, ki ga želimo izboljšati. V našem primeru je to graf $L_n =: G$. Pri vsakem koraku z double edge swap generiramo novega soseda $G'$ in izračunamo $\Delta E = E(G') - E(G)$.
Če je $\Delta E \le 0$, rešitev sprejmemo. V nasprotnem primeru jo sprejmemo z verjetnostjo $$p = \exp\!\left(-\frac{\Delta E}{T}\right),$$
kjer $T>0$ predstavlja trenutni temperaturni parameter. 
S tem omogočimo kontrolirano sprejemanje slabših rešitev zgodaj v 
postopku, kar preprečuje prezgodnjo ujetost v lokalne minimume. Če rešitev sprejmemo nastavimo $G := G'$ in postopek ponavljamo.\\

V našem primeru smo naredili 20000 ponovitev postopka, začetni parameter $T_0$ pa je bil nastavljen tako, da se je v zgodnjih ponovitvah slabša rešitev sprejela z verjetnostjo približno 40 \%, proti koncu postopka pa skoraj nikoli.

\subsection{Rezultati}

Z metodo \emph{simulated annealing} smo za $n \in \{10, 12, \dots, 30\}$ iskali grafe z manjšim številom podpoti od grafa $L_n$.

Kot pričakovano za $n \in \{10, 12, 14\}$ protiprimerov nismo našli, medtem ko smo za vse večje $n$ našli grafe s precej nižjim številom podpoti. Rezultati so prikazani v tabeli.

\begin{table}[h]
\centering
\begin{tabular}{cccc}
\hline
n & $subpath\_number(L_n)$ & boljši $subpath\_number$ & razlika \\
\hline
10 & 1276 & 1276 & 0 \\
12 & 3076 & 3076 & 0 \\
14 & 5504 & 5504 & 0 \\
16 & 12744 & 3640 & 9104 \\
18 & 22532 & 7072 & 15460 \\
20 & 51532 & 12816 & 38716 \\
22 & 90760 & 7156 & 83604 \\
24 & 206800 & 12220 & 194580 \\
26 & 363788 & 21760 & 342028 \\
28 & 827988 & 29568 & 798420 \\
30 & 1456016 & 18520 & 1437496 \\
\hline
\end{tabular}
\caption{Rezultati SA.}
\label{tab:counts}
\end{table}



\begin{figure}[h]
\centering
\includegraphics[width=0.9\textwidth]{slike/SA_grafi.png}
\caption{Grafi pridobljeni s SA za $n =$ 26, 28 in $30$.}
\label{fig:SA_grafi}
\end{figure}


\end{document}