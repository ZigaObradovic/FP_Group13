\documentclass[a4paper,12pt]{article}
\usepackage[utf8]{inputenc}
\usepackage[T1]{fontenc}
\usepackage[slovene]{babel}
\usepackage{lmodern} 
\usepackage{amsmath,amssymb}
\usepackage{graphicx}
\usepackage[a4paper,margin=2cm]{geometry}

\title{\textbf{Group 13: Subpath number - minimal cubic graphs}}
\author{Lovro Levačić, Žiga Obradović}
\date{November 2025}

\begin{document}
\maketitle
\begin{center}
    \large\textbf{KRATEK OPIS}
\end{center}

\section{Uvod}

V projektu obravnavamo domnevo, ki govori o minimizaciji števila poti v razredu 
\textbf{kubičnih grafov}, torej grafov, kjer ima vsako vozlišče stopnjo 3.
Za dani povezani graf $G$ definiramo \textit{subpath number}, označen s $pn(G)$,
kot število vseh preprostih poti v grafu, vključno s trivialnimi potmi dolžine 0.
Preprosta pot je zaporedje vozlišč $(v_0, v_1, \dots, v_\ell)$ brez ponovitev,
kjer je vsak par zaporednih vozlišč povezan z robom.

V literaturi je bila postavljena domneva, da za vsako sodo število vozlišč $n$
obstaja natanko en kubični graf $L_n$, ki ima najmanjši subpath number med vsemi
kubičnimi grafi z $n \geq 10$ vozlišči.
Grafi $L_n$ so sestavljeni iz več kopij grafa $K_4 - e$
(popolnega grafa na štirih vozliščih, iz katerega odstranimo en rob),
ki jih povežemo v verižni strukturi. Na obeh koncih se tej verigi dodata še
posebna \textit{pendant bloka} - eden na 5, drugi pa na 7 vozlišč.

% \begin{figure}[h!]
%     \centering
%     \includegraphics[width=0.7\textwidth]{Ln_graf.png}
%     \caption{$L_n$ grafa za $n = 18$ in $20$}
%     \label{fig:Ln}
% \end{figure}

Kasneje se je pokazalo, da domneva za dovolj velika $n$ ne drži,
vendar še ni znano, pri katerem najmanjšem $n$ se pojavi prva protimera.
Namen našega projekta je torej:
\begin{itemize}
    \item preveriti domnevo za manjša, obvladljiva števila vozlišč $n$,
    \item in poskusiti poiskati protimero, torej kubični graf z manjšim
    subpath number kot $L_n$.
\end{itemize}

\section{Priprava}

Za čim bolj učinkovito delo bova najprej definirala osnovne funkcije, s katerimi bova izdelovala grafe in štela njihove poti:

\begin{enumerate}
  \item $subpath\_number(Graf)$: Funkcija, ki sprejme katerikoli graf in vrne \textbf{subpath number}.
  \item $Ln\_graph(n)$: Funkcija, ki sprejme število vozlišč $n$ in vrne graf $L_n$. Ideja funkcije je, da definiramo tri različne gradnike, ki jih sestavimo skupaj gleda na število $n$.
  \begin{figure}[h!]
    \centering
    \includegraphics[width=0.7\textwidth]{gradniki.png}
    \caption{Gradniki grafa $L_n$}
    \label{fig:gradniki}
\end{figure}
  \item $cubic\_graphs(n)$: Funkcija, ki sprejme število vozlišč $n$ in generira vse možne kubične grafe na $n$ vozliščih (za sode $n \geq 4$).
\end{enumerate}
Z uporabo teh treh funkcij želiva pokriti predvsem prvi del naloge, ki je dokazati domnevo, da ima graf $L_n$ najmanjši subpath number za majhne $n$ in poiskati kakšen protiprimer za večje $n$.

\section{Načrt dela}

Ko bodo osnovne funkcije definirane, bo najina prva naloga ta, da testirava njihovo pravilnost. Funkcijo $subpath\_number$ bova tako najprej testirala na zelo enostavnih grafih, ki niso nujno kubični. Želiva torej, da so testni grafi taki, da subpath number lahko izračunava sama brez dodatnih orodij ("na roke"). Če se bodo dobljeni rezultati ujemali z vrednostmi funkcije, bova funkcijo lahko uporabila tudi za grafe z več vozlišči. Podoben test bova naredila tudi za drugi dve osnovni funkciji.

Naslednji korak pa je primerjati subpath number grafov $L_n$ z vsemi obstoječimi kubičnimi grafi za isti $n$. Zanima naju torej kateri je najmanjši $n$, da obstaja kakšen kubični graf, ki ima subpath number manjši kot $L_n$. Iz protiprimerov, ki jih bova našla pa si želiva konstruirati grafe, ki bodo imeli manjši subpath number za vsak večji $n$.

\section{Hipoteza}

Po prvem testu se zdi, da imajo za večje $n$ naslednji grafi nižji subpath number kot grafi $L_n$:
\begin{itemize}
  \item Če je $n = 6q - 2$ je graf sestavljen iz $q$ pendant blokov na 5 vozliščih, ki so pripeti na $q - 2$ med seboj povezanih vozlišč
  \item Če je $n = 6q$ je graf sestavljen iz $q - 1$ pendant blokov na 5 vozliščih in enega na 7 vozliščih, ki so pripeti na $q - 2$ med seboj povezanih vozlišč
  \item Če je $n = 6q + 2$ je graf sestavljen iz $q - 2$ pendant blokov na 5 vozliščih in dveh na 7 vozliščih, ki so pripeti na $q - 2$ med seboj povezanih vozlišč
\end{itemize}

\begin{figure}[h!]
\centering
\includegraphics[width=0.3\textwidth]{graf_n28.png}
\includegraphics[width=0.3\textwidth]{graf_n30.png}
\includegraphics[width=0.3\textwidth]{graf_n32.png}
\caption{Primeri boljših grafov za $n = 28$, $30$ in $32$}
\end{figure}

Najina naloga sedaj je preveriti resničnost te hipoteze in morda poiskati kakšne grafe z še nižjim subpath number od teh.



\end{document}
