\documentclass[a4paper,12pt]{article}
\usepackage[utf8]{inputenc}
\usepackage[T1]{fontenc}
\usepackage[slovene]{babel}
\usepackage{lmodern} 
\usepackage{amsmath,amssymb}
\usepackage{graphicx}
\usepackage[a4paper,margin=2cm]{geometry}

\title{\textbf{Group 13: Subpath number - minimal cubic graphs}}
\author{Lovro Levačić, Žiga Obradović}
\date{November 2025}

\begin{document}
\maketitle
\begin{center}
    \large\textbf{KRATEK OPIS}
\end{center}

Moj predlog je, da razdeliva to kratko predstavitev na:

\begin{itemize}
  \item \textbf{Uvod:} Na kratko razloživa najin problem.
  \item \textbf{Priprava:} Naštejeva oz. poveva, kako bova naredila osnovne funkcije, ki nama bodo pomagale naprej.
  \item \textbf{Načrt dela:} Na hitro razloživa, kako bova stvari poganjala oz. iskala, kar zahteva naloga.
\end{itemize}

\section{Uvod}

V projektu obravnavamo domnevo, ki govori o minimizaciji števila poti v razredu 
\textbf{kubičnih grafov}, torej grafov, kjer ima vsako vozlišče stopnjo 3.
Za dani povezani graf $G$ definiramo \textit{subpath number}, označen s $pn(G)$,
kot število vseh preprostih poti v grafu, vključno s trivialnimi potmi dolžine 0.
Preprosta pot je zaporedje vozlišč $(v_0, v_1, \dots, v_\ell)$ brez ponovitev,
kjer je vsak par zaporednih vozlišč povezan z robom.

V literaturi je bila postavljena domneva, da za vsako sodo število vozlišč $n$
obstaja natanko en kubični graf $L_n$, ki ima najmanjši subpath number med vsemi
kubičnimi grafi z $n \geq 10$ vozlišči.
Grafi $L_n$ so sestavljeni iz več kopij grafa $K_4 - e$
(popolnega grafa na štirih vozliščih, iz katerega odstranimo en rob),
ki jih povežemo v verižni strukturi. Na obeh koncih se tej verigi dodata še
posebna \textit{pendant bloka} - eden na 5, drugi pa na 7 vozlišč.

\begin{figure}[h!]
    \centering
    \includegraphics[width=0.7\textwidth]{Ln_graf.png}
    \caption{$L_n$ grafa za $n = 18$ in $20$}
    \label{fig:Ln}
\end{figure}

Kasneje se je pokazalo, da domneva za dovolj velika $n$ ne drži,
vendar še ni znano, pri katerem najmanjšem $n$ se pojavi prva protimera.
Namen našega projekta je torej:
\begin{itemize}
    \item preveriti domnevo za manjša, obvladljiva števila vozlišč $n$,
    \item in poskusiti poiskati protimero, torej kubični graf z manjšim
    subpath number kot $L_n$.
\end{itemize}

\section{Priprava}

Za čim bolj učinkovito delo bova najprej definirala osnovne funkcije, s katerimi bova izdelovala grafe in štela njihove poti:

\begin{enumerate}
  \item $subpath\_number(Graf)$: Funkcija, ki sprejme katerikoli graf in vrne \textbf{subpath number}.
  \item $Ln\_graph(n)$: Funkcija, ki sprejme število vozlišč $n$ in vrne graf $L_n$. Ideja funkcije je, da definiramo tri različne gradnike, ki jih sestavimo skupaj gleda na število $n$.
  \item $cubic\_graphs(n)$: Funkcija, ki sprejme število vozlišč $n$ in generira vse možne kubične grafe na $n$ vozliščih (za sode $n \geq 4$).
\end{enumerate}

\section{Načrt dela}

\end{document}
